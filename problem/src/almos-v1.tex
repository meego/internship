\section{ALMOS version Ghassan}

\begin{itemize}

\item On a un seul kernel pour toutes la machine (les réplica ne
                sont pas considérés comme des noyaux indépendants).

         \item Le noyau fonctionn en mémoire virtuelle

         \item But d'ALMOS à ce moment là: avoir un noyau scalable, ce qui
         implique la prise en compte de la localité des accès mémoire, et ce de
         manière intrinsèque pour le noyau.
        
        \item Le but de Ghassan n'était pas de supporter 1To de mémoire, c'était
        d'avoir un système d'exploitation qui passe à l'échelle des
        architectures many-core. Pour cela, il a introduit les clusters
        managers, qui contiennent et gèrent localement les clusters de la
        machine, et notamment \benumline \item le réplica noyau \item les
        processeurs (les évènements du type interruptions, messages etc\ldots,
                        et l'ordonnancement) \item le gestionnaire mémoire
        (capable d'allouer trois types d'objets: des pages, des objets noyau de
         taille fixe, des structures de données temporaires mais dont la taille
         est relativement petite par rapport à celle d'une page) \eenumline


                        \item ca ca marche, ca permet de gérer 256 clusters de 4
                        processeurs chacun et de manière décentralisée. MAIS ca
                        ne prend absolument pas en compte la mémoire disponible

\item \todo{Problème:} On a des processeurs 32 bits, 1To de mémoire, et un
noyau capable de gérer tous les processeurs de l'architecture. On veut
maintenant pouvoir gérer toute la mémoire.

le noyau fonctionne en espace virtuel. Il peut donc adresser au maximum 4Go de
mémoire. Sur ces 4Go octets, il s'en réserve 2Go pour son fonctionnement, ce qui
ne laisse que 2Go à l'utilisateur. On a donc 1To de mémoire physique, et on ne
peut en utiliser que 2Go pour les applications.  \end{itemize}

