\chapter{Définition de la procédure de recette}
\label{chap:tests}

  \todo{chapitre à faire au propre.}\\

  Nous allons à présent donner les différentes procédures de recettes que nous
  allons utiliser pour valider nos solutions.
  
  \section{Migration de processus mono-thread}

    \begin{itemize}
      \item Pour tester les \texttt{fork()} sans \texttt{exec()} : un programme
        C qui \texttt{fork()} à fond et on fait des printfs régulier des
        clusters d'exécution.
      \item Pour tester la migration en cours d'exécution, on devra faire une
        politique ad-hoc dans la DQDT. On fera en sorte qu'au bout d'un temps
        $T$ un processus sera migré sur un autre cluster. Cette modification
        sera faite dans la DQDT. Pour créer ce processus, on peut reprendre le
        programme donné précedemment en ajoutant un appel à \texttt{exec()}
        après le \texttt{fork()}. Chaque processus affiche périodiquement son
        cluster.
    \end{itemize}


  \section{Le multi-threading}

    \begin{itemize}
      \item On reprendre le programme C précédent mais on ajoute en plus la
        création d'un nombre $N$ de thread par processus.
    \end{itemize}


  \section{Évaluation du coût}

    Avec ce qu'on a dit avant, on a de quoi tester la migration mais pas son
    coût en terme de cycles. On va devoir ajouter des compteurs dans les
    programmes pour avoir une idée de la faisabilité de la solution.
