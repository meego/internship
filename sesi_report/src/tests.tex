\chapter{Définition de la procédure de recette}
\label{chap:tests}

  Nous allons à présent donner les différentes procédures de recette que nous
  allons utiliser pour valider nos solutions. Ces tests sont de simples
  microbenchmark nous permettant de valider le fonctionnement de notre solution.

  Ces expériences seront basées sur le système d'exploitation ALMOS-MK (pour
  \textit{Multi-Kernel}), qui est l'évolution d'ALMOS en multi-noyau. Cette
  version du système est actuellement maintenue par Mohamed
  Karaoui. L'architecture d'exécution d'ALMOS-MK sera l'architecture TSAR 40
  bits basique.\\

  Pour prouver le fonctionnement de la migration mono et multi-thread, nous
  allons modifier le mécanisme de migration présent dans ALMOS pour qu'il agisse
  en fonction de nos besoins de test et non ceux du système. On peut se
  permettre cela car on cherche à prouver la migration et non la politique de
  migration. Néanmoins, ces changement seront effacés lors des tests de la DQDT,
  que nous verrons en section~\ref{sec:dqdt-test}.

  \section{Migration de processus mono-thread}

    \subsection{Migration basique}

      On considère ici le schéma \texttt{fork()/exec()} classique vu au
      chapitre~\ref{chap:sol}. Nous allons dans un premier temps modifier la
      DQDT pour que la migration soit faite \textit{de facto} lors d'un appel à
      \texttt{exec()}, et non plus selon le taux d'utilisation de la
      machine. Ensuite, on lancera un simple programme \texttt{fork()/exec()} où
      le fils affiche le numéro de son cluster et de son c\oe ur avant et après
      le \texttt{exec()}.

    \subsection{Cas du \texttt{fork()} sans \texttt{exec()}}

      Nous voulons ici tester la migration de processus fils n'ayant pas fait
      d'appel à \texttt{exec()}. Nous allons donc écrire un programme créeant
      $N$ fils, puis chacun de ces fils affichera à intervalle régulier le
      numéro de cluster et le numero de c\oe ur sur lequel il s'exécute. La DQDT
      sera modifiée pour que la migration ne commence qu'à partir du moment où
      $N$ sera atteint, et pour que les fils soient tous migrés sur des clusters
      différents.

      On définit $N \in [2, 4, 6, 10, 24]$. Le simulateur étant plus lent qu'une
      exécution réelle, nous ne pouvons pas tester pour des valeurs de $N$ trop
      importantes. Néanmoins, si nous avons suffisamment de temps, nous
      souhaitons lancer une exécution sur 240 clusters. Nous avons évalué la
      durée de l'expérience entre 24 et 36 heures.

    \subsection{Migration en cours d'exécution}

      Enfin, pour tester la migration en cours d'exécution, nous allons modifier
      la DQDT pour qu'au bout d'un certain temps $T$, un processus soit migré
      sur un autre cluster. Le programme exécuté par le processus sera une
      boucle infinie où il affichera à intervalle régulier les informations sur
      son cluster et son c\oe ur d'exécution. On fixe arbitrairement $T = 1s$.

  \section{Support du multi-thread}

    On se place ici dans le cadre des threads d'un même processus. Cela change
    le paradigme de migration puisque l'appel système \texttt{exec()} n'est plus
    impliqué. En effet, les threads d'un processus peuvent être migrés à
    n'importe quel moment de leur exécution.\\

    Un processus va créer $N$ threads, $N \in [2, 4, 6, 10, 24]$. Ici, on va
    modifier la DQDT pour que $N-1$ threads\footnote{Le thread initiateur ne
      sera pas migré.} soient migrés sur un cluster différent des autres au bout
    d'un temps $T = 1s$. Les threads afficheront régulièrement le numéro de
    cluster et du c\oe ur d'exécution.


  \section{Cohérence de données}

    \subsection{Les descripteurs de fichiers ouverts}

      La cohérence des fichiers ouverts pourra être testée selon la procédure
      suivante:
      \begin{itemize}
        \item le processus père ouvre un fichier
        \item il crée son fils via \texttt{fork()} et celui-ci est migré
        \item le père fait un \texttt{read()} de $N$ octets et modifie la tête
          de lecture (on fixe $N = 91$)
        \item le fils, via \texttt{ftell()}\footnote{Fonction non POSIX
          néanmoins supportée par la \texttt{libc}~\citep{vonleitner2001diet}
          d'ALMOS}, obtient la position de la tête de lecture, qui doit être
          égale à $N$\\
      \end{itemize}


    \subsection{Les zones mémoires partagées}

      La cohérence des zones mémoire partagées sera testée selon cette
      procédure:
      \begin{itemize}
        \item un processus $P1$ va demander l'allocation d'une région mémoire de
          4 octets en mode \texttt{SHARED}
        \item $P1$ écrit le nombre $N = 10$ dans cet espace
        \item un processus $P2$ mappe l'espace partagé de $P1$ dans son espace
          virtuel
        \item $P2$ lit la valeur contenue, qui doit être $N = 10$
        \item Le processus $P2$ est migré sur un nouveau cluster
        \item $P2$ change la valeur de $N$ : $N\leftarrow 8$
        \item $P1$ lit la valeur de $N$, qui doit être $N = 8$
      \end{itemize}


    \subsection{La distribution des signaux}

      Pour assurer que les signaux sont distribués à tous les threads d'un
      processus, nous allons définir la procédure suivante. Nous allons lancer
      un programme créant $N$ threads, $N \in [2, 4, 6, 10, 24]$, avec $N-1$
      threads migrés comme précédemment. Chaque thread appliquera un masque sur
      le signal \texttt{SIGUSR1}\footnote{Ce type de signal est réservé pour les
        programmes utilisateurs.} permettant de redéfinir un \textit{handler}
      pour ce dernier. Ce \textit{handler} affichera le numéro du cluster et du
      c\oe ur courant. Ensuite, le thread initiateur enverra un signal
      \texttt{SIGUSR1} au processus auquel il appartient. Les threads migrés
      appellerons alors le \textit{handler} et afficherons leur numéro de
      cluster.


  \section{Tests de performance}
  \label{sec:dqdt-test}

    Nous allons devoir ``stresser'' le système avec de vraies applications, et
    comparer les résultats avec ceux obtenus par
    Ghassan~\citeauthor{almaless2014universite} dans sa thèse. Pour cela, nous
    utiliserons les mêmes benchmarks, FFT, LU et Radix~\citep{bailey1989ffts,
      woo1994performance, blelloch1991comparison}, et nous étudierons les temps
    nécessaires à la prise de décision et la migration.

    FFT implémente une version à une dimension de la transformation de Fourrier
    en six étapes. LU implémente une version dite ``contiguë'' d’une
    factorisation de matrice dense en produit de deux matrices
    triangulaires. Radix implémente un tris de nombres entiers basé sur
    l’utilisation d’un arbre radix. Ces applications sont toutes écrites en C,
    relèvent du domaine HPC et expriment le parallélisme de threads.\\

    Si la DQDT n'est pas totalement opérationnelle à ce stade, nous pourrons
    tout de même tester la performance en sachant que les résultats seront
    mauvais. En effet, ALMOS n'appliquera pas une politique de migration
    optimale.

  \nomenclature{ALMOS-MK}{ALMOS Multi-Kernel}
  \nomenclature{FFT}{Fast Fourrier Transformation}
  \nomenclature{LU}{Lower Upper}
