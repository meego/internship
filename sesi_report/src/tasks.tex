\chapter{Identification des tâches à accomplir}
\label{chap:tasks}


  \section{Processus mono-thread}

    La première étape sera de modifier les \texttt{struct fd\_info\_s} et
    \texttt{struct vfs\_file\_s} gérant la description des fichiers. Nous allons
    mettre en place une des deux tables de hachage présentées au
    chapitre~\ref{chap:sol} pour avoir une nouvelle gestion des
    pointeurs. Précisément, cela nous permettra de retrouver des adresses en
    mémoire après une migration sans avoir besoin de \textit{vrais}
    pointeurs. Nous allons pour cela utiliser les numéros des pages physiques.

    Dans un second temps, nous allons appliquer ce même mécanisme aux régions
    virtuelles partagées. Cette partie est néanmoins optionnelle. Bien
    qu'imposée par la norme POSIX, les régions partagées ne sont pas notre
    priorité. En effet, les processus n'ont par défaut aucune zone partagée. En
    revanche, nous voulons d'abord être capable de gérer les descripteurs de
    fichiers, qui eux sont partagés par défaut : \texttt{stdin, stdout} et
    \texttt{stderr} sont tous les trois des fichiers et sont ouverts par tous
    les processus en permanence.

    Ensuite, nous souhaitons implémenter la fonction de migration qui utilisant
    la DQDT si celle-ci indique qu'une migration doit avoir lieu. Cette fonction
    est spécifique aux processus déjà en cours d'exécution. Nous souhaitons
    également implémenter une fonction permettant de migrer les processus en
    cours de création.
    
    Enfin, nous allons modifier l'appel système \texttt{exec()} pour appeler la
    fonction décrite précédemment.


  \section{Processus multi-thread}

    Notre première étape sera de changer la construction des \texttt{pid} comme
    vu au chapitre~\ref{chap:sol}. Nous devrons ensuite ajouter à la fonction de
    migration un message pour l'envoi du nouveau \texttt{pid} au noyau
    d'origine. Enfin, nous devrons mettre en place un serveur système permettant
    la diffusion des signaux à tous les threads d'un même processus, quelque
    soit leur emplacement sur la machine.

    La seconde étape, la plus délicate, sera l'implémentation du concept des
    processus hybrides. Cette partie s'annonce assez difficile et complexe, en
    particulier la gestion de la table des pages au niveau des threads,
    puisqu'elle touche à la base même d'ALMOS.


  \section{La DQDT}  

    La fonction \texttt{dqdt\_update()} permet de mettre à jour les informations
    sur les taux d'utilisation des processeurs. Cette fonction est appelée
    périodiquement pour assurer une vision correcte de l'utilisation de la
    machine. Nous allons devoir modifier cette fonction selon les techniques
    présentées au chapitre~\ref{chap:sol}, à savoir une méthode générique par
    passage de messages, et une autre plus efficace mais spécifique à notre
    architecture.\\

    Nous pouvons noter que si la DQDT n'est pas opérationnelle, nous pouvons
    quand même valider la migration de processus mono-thread et l'exécution de
    processus multi-thread sur plusieurs clusters. Toutefois, le choix des
    ressources ne sera pas pertinent et les performances seront dégradées, mais
    ce n'est pas important ici. C'est la raison pour laquelle la fonction
    \texttt{dqdt\_update()} est une tâche optionnelle.
