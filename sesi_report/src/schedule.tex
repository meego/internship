\chapter{Échéancier}
\label{chap:sched}

  %% Nous allons découper notre travail selon l'ordre des tâches présenté au
  %% chapitre~\ref{chap:tasks}.

  %% \todo{}

  À compter du 11 Mai 2015, date de la présentation orale, il restera 55 jours
  de stage. Nous devons réaliser les trois tâches présentées dans les chapitre
  précédents, plus une phase de tests. En découpant équitablement le temps
  restant, cela donne environ 13 jours pour chaque tâche. Néanmoins, nous allons
  répartir le temps comme suit.\\

  L'ajout de la migration de processus mono-thread n'est pas une opération
  lourde en soi. Néanmoins, nous allons appréhender le noyau ALMOS lors de sa
  réalisation. C'est pourquoi nous avons découpé équitablement le temps de
  réalisation de cette étape avec celle du multi-threading, qui est elle plus
  compliquée, mais qui se fera en ayant une meilleure connaissance du
  noyau. Ainsi, nous avons fixé une durée de 16 jours par chacune de des
  tâches.\\

  La modification de la DQDT est une opération plus simple que les deux
  précédente. Nous estimons qu'elle peut être réalisée en 10 jours.\\

  Enfin, les tests représentent une partie non négligeable, nous accordons donc
  13 jours à cette étape finale.  Ces derniers sont très importants puisqu'ils
  vont nous permettre de valider notre solution, ou à l'inverse, de montrer que
  la voie choisie n'est pas la bonne.\\

  %% Change space between rows and columns
  \setlength{\tabcolsep}{20pt}
  \renewcommand{\arraystretch}{1.3}
  \begin{table}[h]
    \centering
    \begin{tabular}{@{}llll@{}}
      \toprule
      Étape                & \multicolumn{1}{l}
                            {Début}     & Fin        & Total    \\ \midrule
      support mono-thread  & 11 Mai     & 3 Juin     & 16 jours \\
      support multi-thread & 4 Juin     & 25 Juin    & 16 jours \\
      DQDT                 & 26 Juin    & 9 Juillet  & 10 jours \\
      tests                & 10 Juillet & 30 Juillet & 13 jours \\
                           &            &            &          \\ \hline
      Total                &            &            & 55 jours \\
    \end{tabular}
  \end{table}
  %% Restore default parameters
  \setlength{\tabcolsep}{6pt}
  \renewcommand{\arraystretch}{1.0}
