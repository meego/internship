\begin{titlepage}  

  {\begin{center}\huge\textsf{Université Pierre et Marie Curie}\end{center}}


  \vspace{0.4cm}
  

  {\begin{center}\huge\textsf{Mastère de sciences et technologies}\end{center}}
  
  \vspace{0.4cm}
  
  {\begin{center}\huge\textsc{Mention Informatique \\2014 -- 2015} \end{center}}
  
  \vspace{0.4cm}
  
  {\begin{center}\huge\textsf{Spécialité : SAR}\end{center}}
  
  {\begin{center}\large\textsc{Systèmes et Applications Répartis }\end{center}}
  
  \vspace{0.4cm}

  {\begin{center}\Huge\textbf{ALMOS : migration de threads pour un multi-noyau
        large échelle}\end{center}}
  
  \vspace{0.4cm}
  
  {\begin{center}\huge\textsc{Rapport de Pré-soutenance}\end{center}}
  
  \vspace{0.4cm}
  
  {\begin{center}\large\textsf {11 Mai 2015 }\end{center}}
  
  \vspace{0.4cm}
  
  {\begin{center}\Large\textsc{Présenté Par}\end{center}}
  
  {\begin{center}\huge\textsc{Pierre-Yves PÉNEAU}\end{center}}
  
  \vspace{0.4cm}
  
  {\begin{center}\Large\textsc{Encadrants}\end{center}}
  
  {\begin{center}\huge\textsc{Franck Wajsbürt}\end{center}}
  
  {\begin{center}\huge\textsc{Mohamed KARAOUI}\end{center}}
  
  {\begin{center}\large\textsf{Laboratoire d'accueil : LIP6 }\end{center}}
  
  {\begin{center}\large\textsf{Equipe ALSOC}\end{center}}
  
\end{titlepage}
