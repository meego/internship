\chapter{Introduction}

  \hspace{1cm}Au milieu des années 2000, les fabricants de processeurs ont
  atteint une limite technique.  Au-delà de 100 Watts par boitier, il est
  difficile de refroidir les circuits à l'aide de simples ventilateurs. Les
  technologies comme le water cooling~\citep{googleXXXXdatacenters} sont
  coûteuses et énergivores.  Pour continuer l'augmentation de puissance des
  processeurs en profitant de la loi de Moore, ils ont dû cesser de complexifier
  l'architecture des c\oe urs et d'augmenter leur fréquence de
  fonctionnement. Au contraire, ils ont simplifié les c\oe urs pour en mettre
  plusieurs par processeur.  De nos jours, les architectures à 8 c\oe urs sont
  courantes, celles à une cinquantaine de c\oe urs sont disponibles, et d'autres
  comportant plusieurs centaines, voire milliers, sont à prévoir.
  L'augmentation du nombre de c\oe urs par processeur permet l'augmentation du
  nombre d'instructions exécutées par cycle.  Cela impose une augmentation de la
  quantité de mémoire nécessaire et du débit des accès~\citep{hp2012z820,
    puget2013z9pe}. Les systèmes d'exploitation doivent s'adapter pour gérer
  efficacement ces nouvelles ressources.\\

  \hspace{1cm}Dans ce stage, l'architecture matérielle considérée est
  l'architecture TSAR~\citep{greiner2009tsar} développée au LIP6. TSAR est une
  architecture NUMA (\textit{Non Uniform Memory Acces}) à mémoire partagée
  cohérente, composée de 1024 c\oe urs 32 bits et 1To de mémoire physique (40
  bits).  Les c\oe urs sont répartis en clusters contenant chacun 4 c\oe urs et
  gérant un segment de 4Go de mémoire physique. Le choix de c\oe urs 32 bits, et
  non 64bits, est assumé. C'est, selon les architectes, le meilleur compromis en
  énergie dissipée par instruction et cela permet un meilleur usage des caches
  car les pointeurs sont plus petits.  Un système d'exploitation nommé
  ALMOS~\citep{almaless2011almos} a été spécialement développé pour TSAR. Ce
  système est basé sur un noyau monolithique, tout comme Linux ou *BSD. ALMOS
  signifie \textit{Advanced Locality Management Operating System}. En effet, son
  but premier est le placement efficace des données dans les segments mémoires,
  et des threads accédant à ces données sur les c\oe urs.\\

  \hspace{1cm}L'architecture TSAR utilisée lors du développement d'ALMOS n'était
  pas finalisée. Elle ne proposait que 4Go de mémoire physique (32 bits). Elle
  gère désormais 1To (40 bits). Le but de ce stage est de faire évoluer ALMOS
  pour permettre la gestion de ce tera octet.\\

  \hspace{1cm}Nous faisons face à plusieurs problèmes. Le premier est que les
  c\oe urs 32 bits sont limités à 4Go d'espace adressable virtuel. Le noyau doit
  gérer un espace mémoire physique supérieur à l'espace virtuel des
  processeurs. Pour résoudre ce problème, nous verrons que nous allons devoir
  répartir et souvent répliquer toutes les structures du noyau dans chaque
  cluster. \todo{ALMOS, dans sa version 32 bits, a déjà une organisation
    clusterisée pour gérer le co-placement des threads et des données, mais cela
    n'interdit pas d'accéder facilement à toute la mémoire, c'est désormais
    difficile.} Le deuxième problème est donc de revoir la répartition ou
  réplication des structures du noyau et leur mode d'accès. ALMOS va ainsi
  évoluer vers une structure semblable par certains aspects au
  multi-noyau~\citep{baumann2009multikernel}. Ainsi, le troisième problème est
  une conséquence du deuxième. Si certaines structures sont répliquées, il faut
  que le système en garantisse la cohérence et ainsi offre à l'utilisateur
  l'illusion d'un noyau monolithique. \\

  \hspace{1cm}Pour ce stage, nous allons nous concentrer sur les structures
  partagées par les threads d'un processus, telles que les descripteurs de
  fichiers ou les zones mémoires partagées. \\


  \hspace{1cm}Ce document est organisé de la manière suivante: dans le
  chapitre~\ref{chap:subject}, nous présenterons le sujet du stage et la
  problématique. Le chapitre~\ref{chap:sol} expliquera la solution envisagée
  pour ce répondre à la problématique. Ensuite, nous donnerons dans le
  chapitre~\ref{chap:tasks} le découpage des tâches identifié. Enfin, nous
  donnerons dans le chapitre~\ref{chap:tests} la procédure de recette qui sera
  utilisée pour valider notre solution, et dans le chapitre~\ref{chap:sched}
  l'échéancier retenu pour ce travail.
