\chapter{Conclusion}
\label{sec:conclusion}

  Dans ce document, nous avons présenté notre problématique générale ainsi qu'un
  état de l'art des différentes solutions existantes. Nous avons vu que des
  projet comme Popcorn Linux ou Barrelfish, pourtant prometteurs, ont choisit la
  même voie que le noyau Linux: l'avenir des machines repose sur les
  architectures 64 bits. Nous prenons la direction inverse en affirmant que les
  architectures 32 bits peuvent être capables d'offrir la même puissance tout en
  conservant les avantages de la basse consommation et de la faible quantité de
  silicium occupée sur la puce.\\

  L'architecture TSAR du LIP6 en est un bon exemple. Composée de 1024 c\oe urs
  32 bits et de 1To de mémoire, elle est la preuve que le passage à l'échelle du
  matériel est possible. Nous devons à présent adapter les systèmes pour
  profiter de ces possibilités matérielles. C'est le but du système expérimental
  ALMOS, qui vise à exploiter au maximum les performances de tels
  processeurs. Bien que spécifique à TSAR, il tend à montrer que nous pouvons
  construire des architectures 32 bits performantes, moins chères, ayant une
  faible consommation électrique, et pouvoir les exploiter à leur juste valeur
  grâce à une évolution profonde des systèmes d'exploitation.\\

  Néanmoins, l'existence de telles différences entre les possibilités
  matérielles et logicielles depuis une vingtaine d'années posent la question du
  dialogue entre les architectes système et matériel~\citep{mogul2011mind}.
