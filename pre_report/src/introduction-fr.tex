\section{Introduction}

  \hspace{1cm}Depuis le début des années 2000, une des plus anciennes loi de
  l'informatique commence à montrer ses limites: la loi de Moore. En effet, les
  constructeurs de matériels ont cessé d'augmenter la fréquence des processeurs,
  principalement pour des raisons énergétiques et non techniques, et ont fait le
  choix d'augmenter le nombre de c\oe urs par processeur, ainsi que le nombre de
  processeurs. De nos jours, des architectures avec une cinquantaine de c\oe urs
  sont courantes, et d'autre comportants plusieurs centaines voire milliers de
  c\oe eurs sont à prévoir.\newline
  
  \hspace{1cm}Ces architectures représentent un défi de taille pour les systèmes
  d'exploitation. En effet, elle adressent la problématique de la gestion d'une
  multitude de ressources hétérogènes de manière performante. Les unités de
  traitement en sont un bon exemple: on dispose aujourd'hui de CPU (CISC ou
  RISC) de FPU, et de GPU. Tous les trois sont des unités d'exécution
  d'instruction, mais sont radicalement différents dans leur mode de
  fonctionnement. Pourtant, les architectures massivement multi-c\oe urs
  embarquent tout ces différents équipements, faisant de leur gestion un élément
  central pour un système d'exploitation.  Un second autre problème soulevé par
  ces architectures est la gestion de mémoire, de plus en plus grande. Il a
  majoritairement été résolu par l'avènement des processeurs 64 bits, ou par les
  mécanismes d'extension d'adresses sur les processeurs 32 bits. Néanmoins, nous
  nous trouvons dans le cdare dans ce stage dans une situation ou nous ne
  pouvons utiliser aucune de ces deux solutions.\newline
  
  \hspace{1cm}Ce document est organisé de la manière suivante: en
  section~\ref{sec:scalability} nous présentons les différents travaux de
  recherche adressant la problématique du passage à l'échelle des systèmes
  d'exploitation. La section~\ref{sec:consistency} est relative aux problèmes de
  cohérence mémoire pour les structures de données dans ces noyaux large
  échelle. En section~\ref{sec:memory} nous nous intéressons à la gestion de
  grande quantité de mémoire sur des architectures 32 bits. La présentation du
  contexte de travail sera fait en section~\ref{sec:context}. Enfin nous
  concluons ensuite en section~\ref{sec:conclusion}.
