\documentclass[tikz,border=10pt]{standalone}
\usepackage{forest}
\usetikzlibrary{arrows.meta, shapes.geometric, calc, shadows}

\colorlet{mygreen}{green!75!black}
\colorlet{col1in}{red!30}
\colorlet{col1out}{red!40}
\colorlet{col2in}{mygreen!40}
\colorlet{col2out}{mygreen!50}
\colorlet{col3in}{blue!30}
\colorlet{col3out}{blue!40}
\colorlet{col4in}{mygreen!20}
\colorlet{col4out}{mygreen!30}
\colorlet{col5in}{blue!10}
\colorlet{col5out}{blue!20}
\colorlet{col6in}{blue!20}
\colorlet{col6out}{blue!30}
\colorlet{col7out}{orange}
\colorlet{col7in}{orange!50}
\colorlet{col8out}{orange!40}
\colorlet{col8in}{orange!20}
\colorlet{linecol}{blue!60}

\begin{document}

\pgfkeys{/forest,
  rect/.append style   = {rectangle, rounded corners = 2pt,
                         inner color = col6in, outer color = col6out},
  ellip/.append style  = {ellipse, inner color = col5in,
                          outer color = col5out},
  orect/.append style  = {rect, font = \sffamily\bfseries\LARGE,
                         text width = 325pt, text centered,
                         minimum height = 10pt, outer color = col7out,
                         inner color=col7in},
  oellip/.append style = {ellip, inner color = col8in, outer color = col8out,
                          font = \sffamily\bfseries\large, text centered}}
\begin{forest}
  for tree={
      font=\sffamily\bfseries,
      line width=1pt,
      draw=linecol,
      ellip,
      align=center,
      child anchor=north,
      parent anchor=south,
      drop shadow,
      l sep+=12.5pt,
      edge path={
        \noexpand\path[color=linecol, rounded corners=5pt,
          >={Stealth[length=10pt]}, line width=1pt, ->, \forestoption{edge}]
          (!u.parent anchor) -- +(0,-5pt) -|
          (.child anchor)\forestoption{edge label};
        },
      where level={3}{tier=tier3}{},
      where level={0}{l sep-=15pt}{},
      where level={1}{
        if n={1}{
          edge path={
            \noexpand\path[color=linecol, rounded corners=5pt,
              >={Stealth[length=10pt]}, line width=1pt, ->,
              \forestoption{edge}]
              (!u.west) -| (.child anchor)\forestoption{edge label};
            },
        }{
          edge path={
            \noexpand\path[color=linecol, rounded corners=5pt,
              >={Stealth[length=10pt]}, line width=1pt, ->,
              \forestoption{edge}]
              (!u.east) -| (.child anchor)\forestoption{edge label};
            },
        }
      }{},
  }
  [Processus\\\texttt{struct task}, inner color=col1in, outer color=col1out
    [Virtual Memory\\Manager\\\texttt{struct vmm\_s},   inner color=col2in, outer color=col2out
      [Physical Memory\\Manager\\\texttt{struct pmm\_s}, inner color=col2in, outer color=col2out
	[Page level\\\texttt{struct page\_s}, inner color=col2in, outer color=col2out]
	[Page operators\\\texttt{struct ppn\_s}, inner color=col2in, outer color=col2out]
	[Cluster level\\\texttt{struct cluster\_s}, inner color=col2in, outer color=col2out]
      ]
      [Virtual Region Abstract\\Layer\\\texttt{struct vm\_region\_s}, inner color=col4in, outer color=col4out
	      [Virtual zones\\(heap and private section)\\\texttt{vm\_region\_s}, inner color=col4in, outer color=col4out]
      ]
    ]
    [Reconstruction\\Algorithms, inner color=col3in, outer color=col3out
      [Convex Relaxation
        [Sparse Signal\\Estimate, rect, name=sse1
        ]
      ]
      [Greedy Pursuits
        [Sparse Signal\\Estimate, rect, name=sse2
        ]
      ]
      [, phantom, calign with current
        [A\\B, phantom
          [Our Work, orect, name=us
            [{Improved Sparse Signal Estimate!}, oellip
            ]
          ]
        ]
      ]
      [Non-Convex\\Minimisation Methods
        [Sparse Signal\\Estimate, rect, name=sse3
        ]
      ]
      [Combinatorial\\Algorithms
        [Sparse Signal\\Estimate, rect, name=sse4
        ]
      ]
    ]
  ]
  \begin{scope}[color = linecol, rounded corners = 5pt,
    >={Stealth[length=10pt]}, line width=1pt, ->]
    \draw (sse2.south) -- (us.north -| sse2.south);
    \draw (sse3.south) -- (us.north -| sse3.south);
    \coordinate (c1) at ($(sse1.south)!2/5!(sse2.south)$);
    \coordinate (c2) at ($(sse3.south)!2/5!(sse4.south)$);
    \draw (sse1.south) -- +(0,-10pt) -| (us.north -| c1);
    \draw (sse4.south) -- +(0,-10pt) -| (us.north -| c2);
  \end{scope}
\end{forest}
\end{document}
