\section{Problèmes} \label{sec:problem}

  À présent, nous allons voir quels sont les problèmes apportés par les choix
  architecturaux, tant matériels que logiciels.

  \subsection{Partage de structures}

    Comme nous l'avons vu en section~\ref{sec:fork}, un processus père partage
    avec son fils différentes structures. Les \texttt{struct vfs\_file\_s}
    (l'équivalent ALMOS des \texttt{struct file} de Linux) par exemple, mais
    aussi les régions virtuelles crées en mode \textbf{shared}, que l'on utilise
    dans les IPC~\cite{ipc}. Cette structure doivent se retrouver dans chaque
    fils crée depuis ce père, et doivent en permanence être cohérente entre tout
    ces processus. Chaque modification de l'un d'entre eux et visibles par les
    autres. Dans un contexte de machine non clusterisée, cette cohérence est
    implicite puisque la mémoire n'est pas découpée entre clusters. On profite
    ainsi de pouvoir accéder à n'importe quelle adresse mémoire (sauf les
    adresses noyau) depuis n'importe quel processus. La cohérence est donc
    intrinsèquement garantie, grâce au modèle architectural de la machine. Dans
    le cas de ALMOS, cette propriété n'est pas garantie.


  \subsection{Architecture clusterisée}

    On rappelle ici que l'architecture de travail est une architecture cc-NUMA
    clusterisée, et que ces derniers sont autonomes. Ils exécutent chacun le
    code noyau, chacun possédant sa pile d'exécution, ses données, et surtout
    son banc mémoire de 4Go. Ainsi, dans le cas d'une création à distance, il
    faudra répliquer les structures de données du père dans la mémoire du
    cluster où sera placée le fils. Cela nous amène à deux problèmes conséquents
    \benumline \item cette réplication aura un surcoût mémoire potentiellement
    important, du aux échanges de messages entre les clusters pour l'allocation
  de la mémoire et la copie des informations \item il faudra obligatoirement
    mettre en place un mécanisme de maintient de la cohérence entre ces deux
  structures représentant la même chose mais étant à deux endroits différents en
mémoire physique \item il faudra que ce mécanisme de cohérence passe à
  l'échelle, à minima, de l'architecutre TSA, puisque c'est cette dernière qui
  est considérée comme étant la plateforme de base pour exécuter
  ALMOS\eenumline.


  \subsection{}
