\begin{paragraph}{Remarque:}

  \textit{On suppose ici que le lecteur possède un minimum de connaissance sur
          les systèmes d'exploitation. Il est en particulier nécessaire de
                  comprendre ce qu'est un \textbf{appel système}, un
                  \textbf{processus}, une \textbf{table des pages} et un
                  \textbf{descripteur de fichier}. Il faut également avoir lu le
                  papier de Ghassan Almaless sur ALMOS\cite{almos-cfse}. Avoir
                  lu la thèse de Ghassan Almaless est évidemment
                  conseillé\cite{almos-phd}.}

\end{paragraph}


\section{Introduction}

  Cette documentation est relative à la version 40 bits de
  ALMOS\footnote{Version utilisée lors de la rédaction de cette documentation:
          \textbf{commit \#662 sur la branche de Mohamed}}.

  On s'intéresse ici à la notion de processus, de threads, aux services rendus
  par la table des pages et au fonctionnement de l'appel système
  \texttt{fork()}. Cet appel système, normalisé POSIX\cite{posix}, est la seule
  et unique manière de créer des processus sur un système UNIX. Il est en
  général suivi par un appel à l'une des fonction de la famille \texttt{exec()}.

  On rappelle que l'on considère l'architecture TSAR\cite{tsar} comme étant
  l'architecture matérielle sous-jacente à notre noyau. De plus, nous utilions
  la version de Mohamed Karaoui compatible avec ALMOS 40 bits.

  \begin{paragraph}{Objectifs}
    \begin{itemize}
  
        \item identifier de manière précise ce qu'est un processus
  
        \item comprendre les mécanismes du \texttt{fork()} et du
        \texttt{exec()\footnote{\todo{09.03.2015: le \texttt{exec()} est mis de
                côté pour l'instant}}}
          
        \item identifier les structures de données manipulées lors d'une telle
        opération
          
        \item comprendre les difficultés relatives à la manipulation de ces
        structures sur une architecture clusterisée comme ALMOS
          
    \end{itemize}
  \end{paragraph}

