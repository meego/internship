\section*{Annexes}

  \subsection*{La structure task\_s}
  
    \lstinputlisting[firstline=47, lastline=97, style=almos,
      caption=\texttt{struct task}.Cette structure représente la notion la plus
      haut-niveau possible pour le système d'exploitation: les tâches qu'il doit
      ordonnancer. Chaque tâche est un ensemble de structure permettant de la
      définir.On trouve principalement tout ce qui concerne les zones mémoire de
      la tâche  sa localisation sur la machine (en terme de numéro de cluster et
      de cpu) et tout ce qui conserne les fichiers tout les threads de cette
    tâche. Lors d'un \texttt{fork()} c'est cette structure qui est dupliquée
  puis modifiée.]{include/code/task.h} \label{lst:task}
    
  \subsection*{La structure fork\_info}

  \lstinputlisting[style=almos, firstline=47, lastline=56,
  caption=\texttt{struct fork\_info\_t}. \label{lst:fork_info_t} Cette structure
est utilisée tout au long du fork pour passer les informations vitales entre
toutes les fonctions.]{include/code/task.c}
    
    \newpage

  \subsection*{Le VFS}

    \subsubsection*{La structure node}

      \lstinputlisting[style=almos, firstline=78, lastline=103,
      caption=\texttt{struct vfs\_node\_s}.]{include/code/vfs.h}
      \label{lst:vfs_node}
  
    \subsubsection*{La structure file}
   
      \lstinputlisting[style=almos, firstline=105, lastline=117,
      caption=\texttt{struct vfs\_file\_s}.]{include/code/vfs.h}
      \label{lst:vfs_file}
  
      %Cette structure est l'équivalent de la \texttt{struct file} du noyau
      %Linux.  Ce niveau d'abstraction permet quant à lui de représenter la
      %manipulation de l'information: \texttt{read(), write()}, les offsets des
      %têtes de lecture\ldots
